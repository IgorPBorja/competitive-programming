\documentclass{article}
\usepackage[utf8]{inputenc}

% imports 
%Prefer 
%\iff, \implies and \impliedby

%Bibliography stuff
%For adding hyperlinks
\usepackage[hyphens]{url} % avoids too large urls by allowing line breaking at hyphens
% (best loaded BEFORE hyperref)

\usepackage{parskip} %avoid margins in the beginning of paragraphs
%Math symbols
\usepackage{amsmath}
\usepackage{amsthm}
\usepackage{amssymb}
%Paper geometry
\usepackage{geometry}
\geometry{a4paper, margin = 1cm}

% CODE LISTING
\usepackage{listings}
\lstnewenvironment{code}[1][C++]
{\lstset{language=C++,numbers=left,numberstyle=\tiny}
  \minipage{\dimexpr\textwidth-2em}
}
{\endminipage}

%For adding images
\usepackage{graphicx}
%Defining floor symbols
\usepackage{mathtools}
\DeclarePairedDelimiter\ceil{\lceil}{\rceil}
\DeclarePairedDelimiter\floor{\lfloor}{\rfloor}
%Fixing parenthesis sizes
% \usepackage{physics} %use \qty before parenthesis % NOT DOWNLOADED YET
%Adjustable margins
\def\changemargin#1#2{\list{}{\rightmargin#2\leftmargin#1}\item[]}
\let\endchangemargin=\endlist %allows you to create an environment \begin{changemargin}{x cm}{y cm} \end{changemargin} - where x is the adjust to left margin and y to the right margin 
\usepackage{changepage}
\newtheorem{theorem}{Teorema}[section] %Delimit "the numbered theorem" environment
%the [section] counter establishes that counting resets every section, i.e second theorem in section 7 will be 7.2. 
\newtheorem{lemma}{Lema}[section]
\newtheorem{corollary}{Corolário}[theorem]
\newtheorem{definition}{Definição}[section]
\newtheorem{example}{Exemplo}[section]
\usepackage{hyperref} %To be able to label lemmas and theorems with \label{} and automate the process of referring to them with \cref{} 
%Names for referrencing
\usepackage{cleveref}
\crefname{lemma}{Lema}{Lemas}
\crefname{theorem}{Teorema}{Teoremas}
\crefname{corollary}{Corolário}{Corolários}
\crefname{definition}{Definição}{Definições}
\crefname{example}{Exemplo}{Exemplos}
\crefname{section}{Seção}{Seções}
\crefname{subsection}{Subseção}{Subseções}

%Proof, solution, claim and case environments
\newenvironment{proofenv}{\begin{adjustwidth}{0.25cm}{0.25cm}\begin{proof}[Prova]}{\end{proof}\end{adjustwidth}}
\newenvironment{solutionenv}{\begin{adjustwidth}{0.25cm}{0.25cm}\begin{proof}[Solução]}{\end{proof}\end{adjustwidth}}
\newenvironment{case}[1][1]{\begin{adjustwidth}{0.25cm}{0.25cm}\textbf{Caso #1: }}{\end{adjustwidth}}
%[1] is the first paramater (case num), [1] is its default value
\newenvironment{claim}[1][1]{\begin{adjustwidth}{0.25cm}{0.25cm}\textbf{Afirmação #1: }}{\end{adjustwidth}}
\newcommand{\vect}[1]{\mathbf{#1}}
\newcommand{\Bij}{\operatorname{Bij}} % default value = empty, for first parameter
\newcommand{\sgn}{\operatorname{sgn}}
\newenvironment{generic}[2]{\begin{adjustwidth}{0.25cm}{0.25cm}\textbf{#1 #2: }}{\end{adjustwidth}}
\newcommand{\inner}[2]{\langle #1, #2 \rangle}
\newcommand{\emptyinner}[0]{\langle , \rangle}
\newcommand{\R}{\mathbb{R}}
\newcommand{\K}{\mathbb{K}}
\newcommand{\N}{\mathbb{N}}

% NOT DOWNLOADED YET
% \usepackage{algorithm}
% \usepackage{algpseudocode}

\begin{document}
\section{Enunciado}
O problema é o \textbf{2023A}, do contest 980 (Div1), da plataforma CodeForces, disponível em \url{https://codeforces.com/problemset/problem/2023/A}

\section{Solução (em C++)}

\begin{code}[C++]
    #define TESTCASES
    #include <bits/stdc++.h>
    using namespace std;
     
    #define i64 int64_t
     
    void solve(){
        i64 n;
        cin >> n;
        pair<i64, i64> a[n];
        for (i64 i = 0; i < n; i++){
            cin >> a[i].first >> a[i].second;
        }
        auto cmp = [](pair<i64, i64> p1, pair<i64, i64> p2){
            return p1.first + p1.second < p2.first + p2.second;
        };
        sort(a, a + n, cmp);
        for (i64 i = 0; i < n; i++){
            cout << a[i].first << " " << a[i].second << " ";
        }
        cout << endl;
    }
     
    signed main(){
        int t;
    #ifdef TESTCASES
        cin >> t;
    #endif
        while (t--){
            solve();
        }
    }
\end{code}  

\section{Demonstração}

Basicamente, o argumento (guloso) que queremos mostrar é que o ideal é ordenar os pares por ordem não-decrescente de soma.
Para tanto, empregamos um argumento de troca: de fato, suponha que uma solução ótima \(S\) (que é simplesmente representada por
uma permutação \(\sigma: \{1, \dotsc, n\} \to \{1, \dotsc, n\}\) dos índices) não segue a ordem não-decrescente de soma dos pares.
Isso significa que a permutação associada a \(S\) da sequência original de somas \((a_{1, 1} + a_{1, 2}, \dotsc, a_{n, 1} + a_{n, 2})\)
possui alguma inversão - e é fato conhecido que, \textbf{se uma permutação possui uma inversão, então existe um par de elementos
adjacentes que forma uma inversão} (isto é, existe um par de elementos adjacentes fora de ordem).

Seja então \(i < n - 1\) tal que \(a_{\sigma(i), 1} + a_{\sigma(i), 2} > a_{\sigma(i + 1), 1} + a_{\sigma(i + 1), 2}\).
Para simplificar notação, denote por \(s(x, y) = (x > y) - (y > x)\) - que é \(1\) quando \(x > y\) e \(-1\) quando \(x < y\),
e denote por \(a_{\sigma(i)}\) por \(\alpha\) e \(a_{\sigma(i + 1)}\) por \(\beta\).

Então, considerando a solução \(S^{*}\) obtida trocando as posições \(i\) e \(i + 1\) temos que o número de inversões
de \(S^{*}\) é igual ao número de inversões de \(S\) mais
\[\Delta = s(\beta_1, \alpha_1) + s(\beta_1, \alpha_2) + s(\beta_2, \alpha_1) + s(\beta_2, \alpha_2)\]

Como as inversões internas (entre dois elementos de um mesmo par) não importam, pois existem independente da ordenação
dos pares em si, podemos supor sem perda de generalidade que \(\alpha_1 \leq \alpha_2\) e \(\beta_1 \leq \beta_2\). Então temos dois casos:

\begin{enumerate}
    \item Se \(\beta_2 > \alpha_2\) então como \(\alpha_1 + \alpha_2 > \beta_1 + \beta_2\) temos que \(\beta_1 < \alpha_1\)
    e portanto a ordem dos elementos é \(\beta_1 < \alpha_1 < \alpha_2 < \beta_2\). Assim:
    \[
        \Delta = -1 + (-1) + 1 + 1 = 0
    \]
    \item Se \(\beta_2 < \alpha_2\) então \(\beta_1 < \alpha_2\) também, e com isso:
    \[
        \Delta = s(\beta_1, \alpha_1) + (-1) + s(\beta_2, \alpha_1) + (-1) \leq -2 + 2 = 0
    \]
    \item Se \(\beta_2 = \alpha_2\) então \(\beta_1 < \alpha_1\) e portanto
    \[
        \Delta = -1 + (-1) + 1 + 0 < 0
    \]
\end{enumerate}
Em todo caso, \(S^{*}\) é melhor ou igual a \(S\) e possui menos inversões no vetor de somas dos pares, mostrando
que ordenar os pares por ordem não-decrescente de soma é ótimo.

\end{document}