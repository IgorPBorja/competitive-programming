\documentclass{article}
\usepackage[utf8]{inputenc}

% imports 
%Prefer 
%\iff, \implies and \impliedby

%Bibliography stuff
%For adding hyperlinks
\usepackage[hyphens]{url} % avoids too large urls by allowing line breaking at hyphens
% (best loaded BEFORE hyperref)

\usepackage{parskip} %avoid margins in the beginning of paragraphs
%Math symbols
\usepackage{amsmath}
\usepackage{amsthm}
\usepackage{amssymb}
%Paper geometry
\usepackage{geometry}
\geometry{a4paper, margin = 1cm}

% CODE LISTING
\usepackage{listings}
\lstnewenvironment{code}[1][C++]
{\lstset{language=C++,numbers=left,numberstyle=\tiny}
  \minipage{\dimexpr\textwidth-2em}
}
{\endminipage}

%For adding images
\usepackage{graphicx}
%Defining floor symbols
\usepackage{mathtools}
\DeclarePairedDelimiter\ceil{\lceil}{\rceil}
\DeclarePairedDelimiter\floor{\lfloor}{\rfloor}
%Fixing parenthesis sizes
% \usepackage{physics} %use \qty before parenthesis % NOT DOWNLOADED YET
%Adjustable margins
\def\changemargin#1#2{\list{}{\rightmargin#2\leftmargin#1}\item[]}
\let\endchangemargin=\endlist %allows you to create an environment \begin{changemargin}{x cm}{y cm} \end{changemargin} - where x is the adjust to left margin and y to the right margin 
\usepackage{changepage}
\newtheorem{theorem}{Teorema}[section] %Delimit "the numbered theorem" environment
%the [section] counter establishes that counting resets every section, i.e second theorem in section 7 will be 7.2. 
\newtheorem{lemma}{Lema}[section]
\newtheorem{corollary}{Corolário}[theorem]
\newtheorem{definition}{Definição}[section]
\newtheorem{example}{Exemplo}[section]
\newtheorem{problem}{Problema}
\usepackage{hyperref} %To be able to label lemmas and theorems with \label{} and automate the process of referring to them with \cref{} 
%Names for referrencing
\usepackage{cleveref}
\crefname{lemma}{Lema}{Lemas}
\crefname{theorem}{Teorema}{Teoremas}
\crefname{corollary}{Corolário}{Corolários}
\crefname{definition}{Definição}{Definições}
\crefname{example}{Exemplo}{Exemplos}
\crefname{problem}{Problema}{Problemasx}
\crefname{section}{Seção}{Seções}
\crefname{subsection}{Subseção}{Subseções}

%Proof, solution, claim and case environments
\newenvironment{proofenv}{\begin{adjustwidth}{0.25cm}{0.25cm}\begin{proof}[Prova]}{\end{proof}\end{adjustwidth}}
\newenvironment{solutionenv}{\begin{adjustwidth}{0.25cm}{0.25cm}\begin{proof}[Solução]}{\end{proof}\end{adjustwidth}}
\newenvironment{case}[1][1]{\begin{adjustwidth}{0.25cm}{0.25cm}\textbf{Caso #1: }}{\end{adjustwidth}}
%[1] is the first paramater (case num), [1] is its default value
\newenvironment{claim}[1][1]{\begin{adjustwidth}{0.25cm}{0.25cm}\textbf{Afirmação #1: }}{\end{adjustwidth}}
\newcommand{\vect}[1]{\mathbf{#1}}
\newcommand{\Bij}{\operatorname{Bij}} % default value = empty, for first parameter
\newcommand{\sgn}{\operatorname{sgn}}
\newenvironment{generic}[2]{\begin{adjustwidth}{0.25cm}{0.25cm}\textbf{#1 #2: }}{\end{adjustwidth}}
\newcommand{\inner}[2]{\langle #1, #2 \rangle}
\newcommand{\emptyinner}[0]{\langle , \rangle}
\newcommand{\R}{\mathbb{R}}
\newcommand{\K}{\mathbb{K}}
\newcommand{\N}{\mathbb{N}}

% NOT DOWNLOADED YET
% \usepackage{algorithm}
% \usepackage{algpseudocode}

\begin{document}
\section{Coordinate compression: introdução}

\section{Um problema do USACO: Silver}
\begin{problem}
    (\url{http://www.usaco.org/index.php?page=viewproblem2&cpid=1063})
    Farmer John's largest pasture can be regarded as a large 2D grid of square "cells" (picture a huge chess board). Currently, there are N
    cows occupying some of these cells (\(1\leq N\leq2500\)).
   Farmer John wants to build a fence that will enclose a rectangular region of cells; the rectangle must be oriented so its sides are parallel with the x and y axes, and it could be as small as a single cell. Please help him count the number of distinct subsets of cows that he can enclose in such a region. Note that the empty subset should be counted as one of these.
   
   \textbf{INPUT FORMAT} (input arrives from the terminal / stdin):
   The first line contains a single integer \(N\). Each of the next \(N\) lines Each of the next \(N\) lines contains two space-separated integers, indicating the \((x,y)\)coordinates of a cow's cell. All x coordinates are distinct from each-other, and all y coordinates are distinct from each-other. All x and y values lie in the range \(0 \dotsc 10^9\).

   \textbf{OUTPUT FORMAT} (print output to the terminal / stdout):
   The number of subsets of cows that FJ can fence off. It can be shown that this quantity fits within a signed 64-bit integer (e.g., a "long long" in C/C++).
\end{problem}

\begin{solutionenv}
Inicialmente, criamos elementos que serão necessários e realizamos a compressão de coordenadas:
\begin{generic}{Definições iniciais}{}

    Utilizamos \textbf{vetores indexados a partir de 1} nessa porção teórica da solução.

    Como \(x, y \leq 10^9\), o que é muito grande para qualquer algoritmo linear ou superior, utilizamos \textbf{compressão de coordenadas}, da seguinte maneira: 
    \begin{itemize}
        \item Criamos uma lista \(x_c[]\) das componentes no eixo \(x\) de cada coordenada, em ordem crescente. É feito em \(\Theta(n log n)\), pois as componentes do eixo \(x\) são distintas para qualquer par de coordenadas.
        \item Definimos \(y_c[]\) de forma análoga para as componentes do eixo \(y\). 
    \end{itemize}


    \begin{lemma}
        A compressão de coordenadas não afeta o resultado: a quantidade de subconjuntos que podem ser cobertos por retângulos na configuração "comprimida" é a mesma que na configuração original.
    \end{lemma}

    Dessa forma, definimos um mapa/função \textbf{monótona} \(f_x: \{x_1, \dotsc, x_n\} \to \{1, \dotsc, n\}\) que "ignora" todas as coordenadas não utilizadas no eixo \(x\) (com \(f_y\) análoga). Assim, definimos a matriz \(A\) de ordem \(n \times n\) da seguinte forma:
    \[
    A[i][j] = \begin{cases}
        1 \text{ se existe ponto em } (x_i, y_j) \\
        0 \text{ do contrário }
    \end{cases}    
    \]
    para todo \(1 \leq i, j \leq n\). Isto é, \(A\) é a nova configuração \textbf{comprimida} de disposição dos nossos pontos.

    Portanto, defina recusivamente \(P_s\) como a seguinte matriz  
    \[
        P_s[i][j] = A[i][j] + P_s[i - 1][j] + P_s[i][j - 1] - P_s[i - 1][j - 1]
    \]
    definindo \(P[i][j] = 0\) quando \(i = 0\) ou \(j = 0\).
    \begin{lemma}\label{2d-prefixsum}
        \(P[i][j]\) é a quantidade de pontos \((x, y)\) com \(1 \leq f_x(x) \leq i\) e \(1 \leq f_y(y) \leq j\).

        Isto é, é a quantidade de pontos na caixa \([1, i] \times [1, j]\) na nova configuração comprimida.
    \end{lemma} 
\end{generic}

\begin{generic}{Resolvendo o problema}{}
    Trabalhamos a partir de agora sempre na configuração comprimida, em que existe exatamente um ponto em cada linha e exatamente um ponto em cada coluna. 

    Seja \(P\) o conjunto de pontos marcados na grade (isto é, a configuração das \(n\) vacas no pasto).

    Com essas definições, podemos partir para a resolução do problema. Para todo subconjunto \(A \subset P\) que pode ser coberto por um retângulo, sem que elementos de \(P\setminus A\) sejam cobertos também, temos 3 casos:
    \begin{itemize}
        \item[0] \(|A| < 2\) (trivial)
        \item[1] O retângulo possui duas extremidades em pontos de \(P\)
        \item[2] \(|A| \geq 2\), porém o retângulo possui menos de duas extremidades em pontos de \(P\).
    \end{itemize}

    Note que é impossível existir um retângulo com 3 ou 4 extremidades correspondendo a pontos marcados, do contrário existiriam pontos marcados com mesma coordenada \(x\) ou \(y\), o que não ocorre segundo o enunciado.

    Além disso, o terceiro caso pode ser obtido expandindo um retângulo que satisfaz o segundo caso para cima ou para baixo.
    \begin{definition}
        Seja \(a: \{1, \dotsc, n\}^3 \to \{1, \dotsc, n\}\) definida da seguinte forma: \(a(x_0, x_1, y_0)\) para \(x_0 \leq x_1\) é a quantidade de pontos \((x, y) \in P\) que satisfazem \(x_0 \leq x \leq x_1\) e \(y > y_0\). Isto é, a quantidade de pontos marcados acima do subretângulo \([x_0, x_1] \times [0, y_0]\).
        
        Defina analogamente \(b: \{1, \dotsc, n\}^3 \to \{1, \dotsc, n\}\) da seguinte forma: \(b(x_0, x_1, y_1)\) para \(x_0 \leq x_1\) é a quantidade de pontos \textbf{abaixo} do subretângulo \([x_0, x_1] \times [y_1, n]\). 
    \end{definition}

    \begin{lemma}
        Suponha que \(p = (x_0, y_0), q = (x_1, y_1)\) são dois pontos marcados na configuração comprimida e sejam \(x_\ell = min(x_0, x_1), \ x_r = \max(x_0, x_1)\), \(y_\ell = min(y_0, y_1),\ y_r = \max(y_0, y_1)\). 

        Então expandindo (ou não) o subretângulo \textbf{apenas para cima e para baixo} \([x_\ell, x_r] \times [y_\ell, y_r]\) podemos cobrir um total de \((a(x_\ell, x_r, y_r) + 1) \cdot (b(x_\ell, x_r, y_\ell) + 1)\) subconjuntos. 
        
        Além disso, cada um desses subconjuntos só pode ser coberto a partir de expansão dessa fonte (desse retângulo com extremidades em \(p, q\)).
    \end{lemma}
    \begin{proofenv}
        Como cada ponto marcado possui coordenada \(y\) distinta de todos os outros, ao expandir para cima podemos cobrir os pontos acima do subretângulo um de cada vez. O mesmo vale para os pontos abaixo do subretângulo.
        
        Logo podemos escolher em cobrir qualquer quantidade \(0 \leq \alpha \leq a(x_\ell, x_r, y_r)\) de pontos acima do subretângulo original (expandindo \(\alpha\) unidades para cima na configuração comprimida), e qualquer quantidade \(0 \leq \beta \leq b(x_\ell, x_r, y_\ell)\) de pontos abaixo. O resultado segue, portanto, pelo princípio multiplicativo. 

        Defina agora \(R(p, q)\) como o retângulo com extremidades em \(p, q\). Quanto à unicidade,  suponha que exista \(p_1, q_1\) distintos de \(p, q\) e um subconjunto \(A \subseteq P\) que pode ser coberto pela expansão vertical de ambos \(R(p, q)\) e \(R(p_1, q_1)\). Defina \(x_\ell', x_r', y_\ell', y_r'\) de forma análoga para \(p_1, q_1\). Como são distintos, temos que algum dos pares \((x_\ell', x_\ell), (x_r', x_r), (y_\ell', y_\ell), (y_r', y_r)\) possui dois valores distintos. Por exemplo, se \(x_\ell' > x_\ell\) e \(x_\ell\) é a coordenada \(x\) de \(p\), então \(R(p_1, q_1)\) nunca pode ser expandido verticalmente de forma a cobrir \(p\). Isso é uma contradição, pois ao expandir verticalmente \(R(p, q)\), sempre continuamos cobrindo \(p\). Todos os outros casos são análogos.
    \end{proofenv}

    \begin{lemma}
        Não é necessário expandir para a esquerda ou para direita. Ou seja: dado um conjunto de pontos \(A \subseteq P\) e dois pontos \(p, q\) marcados, tal que \(A\) pode ser coberto a partir da expansão horizontal de \(R(p, q)\), existem \(p', q'\) marcados tal que \(A\) pode ser coberto da expansão vertical de \(R(p', q')\).
    \end{lemma}
    \begin{proofenv}
        Tome \(p' = (p'_x, p'_y)\) como o ponto mais à esquerda de \(A\) e \(q' = q'_x, q'_y\) como o ponto mais à direita de \(A\), e expanda \(\max_{u \in A}u_y - \max(p'_y, q'_y)\) unidades para cima e \(\min(p'_y, q'_y) - \min_{u \in A}u_y\) unidades para baixo. O conjunto coberto será exatamente \(A\).
    \end{proofenv}

    Portanto, considerando que existem exatamente \(n + 1\) conjunto com cardinalidade menor que \(2\) (os unitários e o vazio), e que todos eles podem ser cobertos por retângulos trivialmente, a resposta final é:
    \begin{equation}\label{main-answer}
        \operatorname{Ans} = n + 1 + \sum_{p, q \in P}
        (
            a(\min(p_x, q_x), \max(p_x, q_x), \max(p_y, q_y)) + 1
        ) \cdot
        (
            b(\min(p_x, q_x), \max(p_x, q_x), \min(p_y, q_y)) + 1
        ) 
    \end{equation}
    
    Porém, pelo \cref{2d-prefixsum}, podemos concluir que 
    \begin{equation}\label{a-transform}
        a(x_0, x_1, y) = (P_s[x_1][n] - P_s[x_0][n]) - (P_s[x_1][y] - P_s[x_0][y])    
    \end{equation}
    pois \(P_s[x_1][n] - P_s[x_0][n]\) conta quantos pontos marcados há em \([x_0, x_1] \times [1, n]\) e \(P_s[x_1][y] - P_s[x_0][y]\) conta quantos pontos marcados há em \([x_0, x_1] \times [1, y]\). Similarmente
    \begin{equation}\label{b-transform}
        b(x_0, x_1, y) = \begin{cases}
            P_s[x_1][y - 1] - P_s[x_0][y - 1] \text{ se } y > 1 \\
            0 \text{ se } y = 1
        \end{cases}
    \end{equation}

    Isso finaliza a corretude do algoritmo a ser demonstrada em código na seção seguinte.
\end{generic}

\begin{proof}[Calculando a complexidade]
    Podemos calcular \(x_c[]\) e \(y_c[]\) em \(O(n \log n)\) com ordenação.
    
    Depois, para cada ponto marcado \(p \in P\), podemos calcular \(f_x(p_x)\) e \(f_y(p_y)\) em \(O(\log n)\) usando busca binária nos vetores \(x_c[], \ y_c[]\), e fazemos a substituição do ponto \(p = (p_x, p_y)\) pelo ponto \textbf{já comprimido} \(p' = (f_x(p_x), f_y(p_y))\). A complexidade total dessa seção novamente é \(O(n \log n)\). Agora, já temos a configuração comprimida.

    Dessa forma, a matriz \(A\) (e consequentemente, a matriz \(P_s\)) podem ser calculadas facilmente em \(O(n^2)\), uma vez que são matrizes \(n \times n\) e cada entrada pode ser determinada em tempo constante. 

    Por fim, usando \cref{a-transform} e \cref{b-transform} para calcular a quantidade de pontos abaixo/acima de um subretângulo em tempo constante, \cref{main-answer} pode ser calculada em \(O(\binom{n}{2}) = O(n^2)\). Isso se deve ao fato de que ela apresenta operações de tempo constante sobre cada par dos \(n\) pontos.

    Como \(\log n = o(n)\), a complexidade final do algoritmo é \(O(n^2)\).
\end{proof}
\end{solutionenv}

\subsection{Código em C++}
\begin{code}
    #include <bits/stdc++.h>
    using namespace std;
    
    int bsearch(int val, int x[], int n)
    {
        int lo = 0, hi = n - 1;
        while (lo <= hi)
        {
            int mid = lo + (hi - lo) / 2;
            if (val == x[mid])
            {
                return mid;
            }
            else if (val < x[mid])
            {
                hi = mid - 1;
            }
            else
            {
                lo = mid + 1;
            }
        }
        return -1;
    }
    
    void swap(int &a, int &b)
    {
        int temp;
        temp = max(a, b);
        a = min(a, b);
        b = temp;
    }
    
    int main()
    {
        int n;
        cin >> n;
        int xs[n], ys[n];
        pair<int, int> points[n];
        for (int i = 0; i < n; i++)
        {
            cin >> points[i].first >> points[i].second;
            xs[i] = points[i].first;
            ys[i] = points[i].second;
        }
        sort(xs, xs + n);
        sort(ys, ys + n);
    
        int ps[n][n];
        for (int i = 0; i < n; i++)
        {
            for (int j = 0; j < n; j++)
            {
                ps[i][j] = 0;
            }
        }
\end{code}

\begin{code}
        // compressing
        for (int i = 0; i < n; i++)
        {
            int ix = bsearch(points[i].first, xs, n);
            int iy = bsearch(points[i].second, ys, n);
            points[i].first = ix;
            points[i].second = iy;
            ++ps[ix][iy];
        }
        for (int i = 0; i < n; i++)
        {
            for (int j = 0; j < n; j++)
            {
                if (j > 0)
                    ps[i][j] += ps[i][j - 1];
                if (i > 0)
                    ps[i][j] += ps[i - 1][j];
                if (i > 0 && j > 0)
                    ps[i][j] -= ps[i - 1][j - 1];
            }
        }
    
        long long ans = 0;
        for (int i = 0; i < n; i++)
        {
            for (int j = 0; j < i; j++)
            {
                int i0 = points[i].first;
                int j0 = points[i].second;
                int i1 = points[j].first;
                int j1 = points[j].second;
                swap(i0, i1);
                swap(j0, j1);
    
                int a = (long long)(ps[i1][n - 1] - ps[i0][n - 1]) - (long long)(ps[i1][j1] - ps[i0][j1]);
                int b = 0;
                if (j0 > 0)
                {
    
                    b += (long long)(ps[i1][j0 - 1] - ps[i0][j0 - 1]);
                }
                ans += (a + 1) * (b + 1);
            }
        }
        cout << ans + n + 1;
    }
\end{code}
\end{document}