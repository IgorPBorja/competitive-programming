\documentclass{article}
\usepackage[utf8]{inputenc}

% imports 
%Prefer 
%\iff, \implies and \impliedby

%Bibliography stuff
%For adding hyperlinks
\usepackage[hyphens]{url} % avoids too large urls by allowing line breaking at hyphens
% (best loaded BEFORE hyperref)

\usepackage{parskip} %avoid margins in the beginning of paragraphs
%Math symbols
\usepackage{amsmath}
\usepackage{amsthm}
\usepackage{amssymb}
%Paper geometry
\usepackage{geometry}
\geometry{a4paper, margin = 1cm}

% CODE LISTING
\usepackage{listings}
\lstnewenvironment{code}[1][C++]
{\lstset{language=C++,numbers=left,numberstyle=\tiny}
  \minipage{\dimexpr\textwidth-2em}
}
{\endminipage}

%For adding images
\usepackage{graphicx}
%Defining floor symbols
\usepackage{mathtools}
\DeclarePairedDelimiter\ceil{\lceil}{\rceil}
\DeclarePairedDelimiter\floor{\lfloor}{\rfloor}
%Fixing parenthesis sizes
% \usepackage{physics} %use \qty before parenthesis % NOT DOWNLOADED YET
%Adjustable margins
\def\changemargin#1#2{\list{}{\rightmargin#2\leftmargin#1}\item[]}
\let\endchangemargin=\endlist %allows you to create an environment \begin{changemargin}{x cm}{y cm} \end{changemargin} - where x is the adjust to left margin and y to the right margin 
\usepackage{changepage}
\newtheorem{theorem}{Teorema}[section] %Delimit "the numbered theorem" environment
%the [section] counter establishes that counting resets every section, i.e second theorem in section 7 will be 7.2. 
\newtheorem{lemma}{Lema}[section]
\newtheorem{corollary}{Corolário}[theorem]
\newtheorem{definition}{Definição}[section]
\newtheorem{example}{Exemplo}[section]
\usepackage{hyperref} %To be able to label lemmas and theorems with \label{} and automate the process of referring to them with \cref{} 
%Names for referrencing
\usepackage{cleveref}
\crefname{lemma}{Lema}{Lemas}
\crefname{theorem}{Teorema}{Teoremas}
\crefname{corollary}{Corolário}{Corolários}
\crefname{definition}{Definição}{Definições}
\crefname{example}{Exemplo}{Exemplos}
\crefname{section}{Seção}{Seções}
\crefname{subsection}{Subseção}{Subseções}

%Proof, solution, claim and case environments
\newenvironment{proofenv}{\begin{adjustwidth}{0.25cm}{0.25cm}\begin{proof}[Prova]}{\end{proof}\end{adjustwidth}}
\newenvironment{solutionenv}{\begin{adjustwidth}{0.25cm}{0.25cm}\begin{proof}[Solução]}{\end{proof}\end{adjustwidth}}
\newenvironment{case}[1][1]{\begin{adjustwidth}{0.25cm}{0.25cm}\textbf{Caso #1: }}{\end{adjustwidth}}
%[1] is the first paramater (case num), [1] is its default value
\newenvironment{claim}[1][1]{\begin{adjustwidth}{0.25cm}{0.25cm}\textbf{Afirmação #1: }}{\end{adjustwidth}}
\newcommand{\vect}[1]{\mathbf{#1}}
\newcommand{\Bij}{\operatorname{Bij}} % default value = empty, for first parameter
\newcommand{\sgn}{\operatorname{sgn}}
\newenvironment{generic}[2]{\begin{adjustwidth}{0.25cm}{0.25cm}\textbf{#1 #2: }}{\end{adjustwidth}}
\newcommand{\inner}[2]{\langle #1, #2 \rangle}
\newcommand{\emptyinner}[0]{\langle , \rangle}
\newcommand{\R}{\mathbb{R}}
\newcommand{\K}{\mathbb{K}}
\newcommand{\N}{\mathbb{N}}

% NOT DOWNLOADED YET
% \usepackage{algorithm}
% \usepackage{algpseudocode}
\newtheorem*{problem*}{Problema}

\begin{document}

\section{O problema}
O problema analisado é o problema D (\emph{Static Range Queries}) do Contest \emph{Usaco Guide Problem Submission}, disponível em \url{https://codeforces.com/gym/102951/problem/D}. 

\fbox{
\begin{minipage}{50em}
    \begin{problem*}
    There is an array \(a\) of length \(10^9\), initially containing all zeroes.

    First perform \(N\) updates of the following form:
    \begin{itemize}
        \item Given integers \(l\), \(r\), and \(v\), add \(v\) to all values \(a_l, \dotsc, a_{r-1}\). 
    \end{itemize}
    Then, answer \(Q\) queries of the following form:
    \begin{itemize}
        \item Given integers \(l\) and \(r\), print the sum \(a_l + \dotsc + a_{r-1}\). 
    \end{itemize}
    
    \hfill\break %% additional line skip
    \textbf{Input}

    Line 1: The two space-separated integers \(N\) and \(Q\)(\(1 \leq N,Q \leq 10^5\))

    Lines \(2\dotsc N+1\): Each line contains three space-separated integers \(l\), \(r\), and \(v\), corresponding to an update (\(0 \leq l < r \leq 10^9,\ v \leq 10^4\)).

    Lines \(N+2\dotsc N+Q+1\): Each line contains two space-separated integers \(l\) and \(r\), corresponding to a query (\(0 \leq l < r \leq 10^9\)).

    \hfill\break %% additional line skip
    \textbf{Output}

    Print \(Q\) lines, the answers to the queries in the same order as given in the input.

    \end{problem*}
\end{minipage}
}
\begin{solutionenv}
Seja \(b\) um vetor de tamanho \(10^9\), contendo inicialmente apenas zeros, e para cada query de update \((l, r, v)\):
\begin{itemize}
    \item Incremente \(b[l]\) por \(v\): \(b[l] \gets b[l] + v\).
    \item Se \(r \neq 10^9\), decremente \(b[r]\) por \(v\): \(b[r] \gets b[r] - v\).
\end{itemize}

Então é fácil mostrar que 
\[a[i] = \sum_{0 \leq j \leq i}b[j]\]
(para todo \(0 \leq i < 10^9\)),isto é, a soma de prefixos de \(b\) resulta no vetor original. 

Sejam \(0 \leq p_0 < \dotsc < p_{k-1} < 10^9\) as posições não-nulas de \(b\), e defina \(p_k = 10^9\). Assim, para todo \(0 \leq i < 10^9\)
\[
\sum_{0 \leq j \leq i}a[j] = \sum_{0 \leq j \leq i}\sum_{0 \leq l  \leq j}b[l] = \sum_{0 \leq l \leq i}(i + 1 - l)b[l]
\] 
pois cada \(b[l]\) aparece uma vez na soma, para cada \(j\) que satisfaz \(l \leq j \leq i\).

Porém, note que \(b\) possui apenas \(O(n)\) posições não-nulas, e \(n << 10^9\). Portanto, podemos realizar \textbf{compressão de coordenadas}

Primeiramente, defina para todo \(0 \leq i \leq 10^9 - 1\)
\begin{align*}
    f(i) = 
    \begin{cases}
        \max\Big(\{j | p_j \leq i\}\Big) \text{ se } i \geq p_0 \\
        -1 \text{, do contrário }
    \end{cases} 
\end{align*} 
É evidente que \(f(i) < k\) (estritamente) pois \(p_k = 10^9 > i\)

Ademais, defina \(x_j := b[p_j]\) e \(y_j = \sum\limits_{0 \leq l \leq j}x_l\) (\(j\)-ésimo prefixo) para todo \(0 \leq j \leq k\). Assim, segue que
\begin{align*}
    \sum_{0 \leq j \leq i}a[j] &= \sum_{0 \leq j \leq f(i)}(i + 1 - p_j)b[p_j] = \sum_{0 \leq j \leq f(i)}(i + 1 - p_j)x_j
    \\
    &= \left(\sum_{0 \leq j \leq f(i)}(p_{f(i)} - p_j)x_j\right) +  (i + 1 - p_{f(i)})\left(\sum_{0 \leq j \leq f(i)}x_j\right)
    \\
    &= \left(\sum_{0 \leq j \leq f(i)}\sum_{j \leq l < f(i)}(p_{l + 1} - p_l)x_j\right) 
    + (i + 1 - p_{f(i)})\left(\sum_{0 \leq j \leq f(i)}x_j\right)
\end{align*}
Assim, podemos notar que, na primeira parcela da soma, cada \(p_{l + 1} - p_l\), para \(0 \leq l < f(i)\), aparece acompanhado por um fator de \(\sum\limits_{0 \leq j \leq l}x_j = y_l\). Logo, invertendo a ordem da soma temos
\begin{align*}
    \sum_{0 \leq j \leq i}a[j] = \sum_{0 \leq l < f(i)}(p_{l + 1} - p_l)y_l + (i + 1 - p_{f(i)})y_{f(i)} = \sum_{0 \leq l \leq f(i)}(p_{l + 1} - p_l)y_l - (p_{f(i) + 1} - i - 1)y_{f(i)}
\end{align*}
Defina \(z_i = \sum\limits_{0 \leq l \leq i}(p_{l+1} - p_l)y_l\). Então
\[\sum_{0 \leq j \leq i}a[j] = z_{f(i)} - y_{f(i)}(p_{f(i) + 1} - i - 1)\]

Portanto, a resposta de uma query \((l, r)\) é 
\[query(l, r) = \sum_{l \leq i < r}a[i] = 
\begin{cases}
    z_{f(r - 1)} - y_{f(r - 1)}(p_{f(r - 1) + 1} - r) \text{ se } l = 0 
    \\
    z_{f(r - 1)} - y_{f(r - 1)}(p_{f(r - 1) + 1} - r) - z_{f(l - 1)} + y_{f(l - 1)}(p_{f(l - 1) + 1} - l) \text{ se } l > 0
\end{cases}\]
\end{solutionenv}

\begin{solutionenv}[Complexidade]
    Podemos calcular os vetores \(p, x\) em \(O(n \log n)\) da seguinte forma:

    \fbox{
    \begin{minipage}{30em}         
        \begin{itemize}
            \item[] Criamos dois vetores de inteiros \(p[], x[]\) vazios.
            \item[] Criamos um vetor de pares de inteiros \(points[]\), inicialmente vazio.
            \item[] Para todo \(0 \leq i < n\)
            \begin{itemize}
                \item[] Lemos uma query de update \(l, r, v\).
                \item[] Adicionamos o par \(\{l, v\}\) a \(points[]\).
                \item[] Adicionamos o par \(\{r, -v\}\) a \(points[]\). 
            \end{itemize}
            \item[] \textbf{Ordenamos} \(points\)  
            \item[] Para todo \(0 \leq i < 2n\)
            \begin{itemize}
                \item[] Se \(p[]\) não é vazio e \(points[i].first\) é igual ao último elemento de \(p[]\), incrementamos o último elemento de \(x[]\) por \(points[i].second\). Do contrário, adicionamos \(points[i].first\) ao final de \(p[]\) e \(points[i].second\) ao final de \(x[]\).
            \end{itemize}
        \end{itemize}
    \end{minipage}
    }

    Ademais, podemos pré-computar os vetores \(y, z\). Por fim, para toda query \((l, r)\), podemos encontrar \(f(l-1)\) e \(f(r-1)\) em \(O(\log |p|) = O(\log n)\) (uma vez que \(|p| \leq 2n\)), e o resto é computado em tempo constante.

    Assim, a complexidade final é \(\boldsymbol{O((n + q)\log n)}\).

\end{solutionenv}

\section{Implementação}
\begin{verbatim}
    #include <bits/stdc++.h>
    using namespace std;
    #define int long long
     
    int query(int l, vector<int>& y, vector<int>& z, vector<int>& p)
    {
        // ans + 1 < |p|
        int lo = 0, hi = p.size() - 2, ans = -1;
        while (lo <= hi)
        {
            int mid = lo + (hi - lo)/2;
            if (p[mid] > l)
            {
                hi = mid - 1;
            } else 
            {
                ans = mid;
                lo = mid + 1;
            }
        }
        if (ans == -1)
        {
            return 0;
        } else 
        {
            return z[ans] - y[ans] * (p[ans + 1] - l - 1); 
        }
    }
     
    signed main()
    {
        int n, q;
        cin >> n >> q;
        pair<int, int> points[2*n];
        
        int l, r , v;
        for (int i = 0; i < n; i++)
        {
            cin >> l >> r >> v;
            points[2*i] = make_pair(l, v);
            points[2*i + 1] = make_pair(r, -v);
        }
        sort(points, points + 2 * n);
        
        vector<int> x, p;
        for (int i = 0; i < 2*n; i++)
        {
            if (p.size() > 0 && points[i].first == p[p.size() - 1])
            {
                x[x.size() - 1] += points[i].second;
            } else 
            {
                p.emplace_back(points[i].first);
                x.emplace_back(points[i].second);
            }
        }
        p.emplace_back(1e9);
        x.emplace_back(0);
        
        for (int i = 1; i < (int)x.size(); i++)
        {
            x[i] += x[i-1];
        }
        vector<int> z((int)x.size() - 1);
        for (int i = 0; i < (int)x.size() - 1; i++)
        {
            z[i] = x[i] * (p[i + 1] - p[i]);
            if (i > 0) z[i] += z[i-1];
        }
        
        for (int i = 0; i < q; i++)
        {
            cin >> l >> r;
            int ans = 0;
            if (r > 0) ans += query(r - 1, x, z, p);
            if (l > 0) ans -= query(l - 1, x, z, p);
            cout << ans << '\n';
        }
    }
\end{verbatim}


\end{document}