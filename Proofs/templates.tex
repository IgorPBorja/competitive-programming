\usepackage[utf8]{inputenc}

% imports 
%Prefer 
%\iff, \implies and \impliedby

%Bibliography stuff
%For adding hyperlinks
\usepackage[hyphens]{url} % avoids too large urls by allowing line breaking at hyphens
% (best loaded BEFORE hyperref)

\usepackage{parskip} %avoid margins in the beginning of paragraphs
%Math symbols
\usepackage{amsmath}
\usepackage{amsthm}
\usepackage{amssymb}
%Paper geometry
\usepackage{geometry}
\geometry{a4paper, margin = 1cm}

% CODE LISTING
\usepackage{listings}
\lstnewenvironment{code}[1][C++]
{\lstset{language=C++,numbers=left,numberstyle=\tiny}
  \minipage{\dimexpr\textwidth-2em}
}
{\endminipage}

%For adding images
\usepackage{graphicx}
%Defining floor symbols
\usepackage{mathtools}
\DeclarePairedDelimiter\ceil{\lceil}{\rceil}
\DeclarePairedDelimiter\floor{\lfloor}{\rfloor}
%Fixing parenthesis sizes
% \usepackage{physics} %use \qty before parenthesis % NOT DOWNLOADED YET
%Adjustable margins
\def\changemargin#1#2{\list{}{\rightmargin#2\leftmargin#1}\item[]}
\let\endchangemargin=\endlist %allows you to create an environment \begin{changemargin}{x cm}{y cm} \end{changemargin} - where x is the adjust to left margin and y to the right margin 
\usepackage{changepage}
\newtheorem{theorem}{Teorema}[section] %Delimit "the numbered theorem" environment
%the [section] counter establishes that counting resets every section, i.e second theorem in section 7 will be 7.2. 
\newtheorem{lemma}{Lema}[section]
\newtheorem{corollary}{Corolário}[theorem]
\newtheorem{definition}{Definição}[section]
\newtheorem{example}{Exemplo}[section]
\newtheorem{problem}{Problema}
\usepackage{hyperref} %To be able to label lemmas and theorems with \label{} and automate the process of referring to them with \cref{} 
%Names for referrencing
\usepackage{cleveref}
\crefname{lemma}{Lema}{Lemas}
\crefname{theorem}{Teorema}{Teoremas}
\crefname{corollary}{Corolário}{Corolários}
\crefname{definition}{Definição}{Definições}
\crefname{example}{Exemplo}{Exemplos}
\crefname{problem}{Problema}{Problemasx}
\crefname{section}{Seção}{Seções}
\crefname{subsection}{Subseção}{Subseções}

%Proof, solution, claim and case environments
\newenvironment{proofenv}{\begin{adjustwidth}{0.25cm}{0.25cm}\begin{proof}[Prova]}{\end{proof}\end{adjustwidth}}
\newenvironment{solutionenv}{\begin{adjustwidth}{0.25cm}{0.25cm}\begin{proof}[Solução]}{\end{proof}\end{adjustwidth}}
\newenvironment{case}[1][1]{\begin{adjustwidth}{0.25cm}{0.25cm}\textbf{Caso #1: }}{\end{adjustwidth}}
%[1] is the first paramater (case num), [1] is its default value
\newenvironment{claim}[1][1]{\begin{adjustwidth}{0.25cm}{0.25cm}\textbf{Afirmação #1: }}{\end{adjustwidth}}
\newcommand{\vect}[1]{\mathbf{#1}}
\newcommand{\Bij}{\operatorname{Bij}} % default value = empty, for first parameter
\newcommand{\sgn}{\operatorname{sgn}}
\newenvironment{generic}[2]{\begin{adjustwidth}{0.25cm}{0.25cm}\textbf{#1 #2: }}{\end{adjustwidth}}
\newcommand{\inner}[2]{\langle #1, #2 \rangle}
\newcommand{\emptyinner}[0]{\langle , \rangle}
\newcommand{\R}{\mathbb{R}}
\newcommand{\K}{\mathbb{K}}
\newcommand{\N}{\mathbb{N}}

% NOT DOWNLOADED YET
% \usepackage{algorithm}
% \usepackage{algpseudocode}