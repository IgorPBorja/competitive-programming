\documentclass{article}
\usepackage[utf8]{inputenc}

% imports 
%Prefer 
%\iff, \implies and \impliedby

%Bibliography stuff
%For adding hyperlinks
\usepackage[hyphens]{url} % avoids too large urls by allowing line breaking at hyphens
% (best loaded BEFORE hyperref)

\usepackage{parskip} %avoid margins in the beginning of paragraphs
%Math symbols
\usepackage{amsmath}
\usepackage{amsthm}
\usepackage{amssymb}
%Paper geometry
\usepackage{geometry}
\geometry{a4paper, margin = 1cm}

% CODE LISTING
\usepackage{listings}
\lstnewenvironment{code}[1][C++]
{\lstset{language=C++,numbers=left,numberstyle=\tiny}
  \minipage{\dimexpr\textwidth-2em}
}
{\endminipage}

%For adding images
\usepackage{graphicx}
%Defining floor symbols
\usepackage{mathtools}
\DeclarePairedDelimiter\ceil{\lceil}{\rceil}
\DeclarePairedDelimiter\floor{\lfloor}{\rfloor}
%Fixing parenthesis sizes
% \usepackage{physics} %use \qty before parenthesis % NOT DOWNLOADED YET
%Adjustable margins
\def\changemargin#1#2{\list{}{\rightmargin#2\leftmargin#1}\item[]}
\let\endchangemargin=\endlist %allows you to create an environment \begin{changemargin}{x cm}{y cm} \end{changemargin} - where x is the adjust to left margin and y to the right margin 
\usepackage{changepage}
\newtheorem{theorem}{Teorema}[section] %Delimit "the numbered theorem" environment
%the [section] counter establishes that counting resets every section, i.e second theorem in section 7 will be 7.2. 
\newtheorem{lemma}{Lema}[section]
\newtheorem{corollary}{Corolário}[theorem]
\newtheorem{definition}{Definição}[section]
\newtheorem{example}{Exemplo}[section]
\usepackage{hyperref} %To be able to label lemmas and theorems with \label{} and automate the process of referring to them with \cref{} 
%Names for referrencing
\usepackage{cleveref}
\crefname{lemma}{Lema}{Lemas}
\crefname{theorem}{Teorema}{Teoremas}
\crefname{corollary}{Corolário}{Corolários}
\crefname{definition}{Definição}{Definições}
\crefname{example}{Exemplo}{Exemplos}
\crefname{section}{Seção}{Seções}
\crefname{subsection}{Subseção}{Subseções}

%Proof, solution, claim and case environments
\newenvironment{proofenv}{\begin{adjustwidth}{0.25cm}{0.25cm}\begin{proof}[Prova]}{\end{proof}\end{adjustwidth}}
\newenvironment{solutionenv}{\begin{adjustwidth}{0.25cm}{0.25cm}\begin{proof}[Solução]}{\end{proof}\end{adjustwidth}}
\newenvironment{case}[1][1]{\begin{adjustwidth}{0.25cm}{0.25cm}\textbf{Caso #1: }}{\end{adjustwidth}}
%[1] is the first paramater (case num), [1] is its default value
\newenvironment{claim}[1][1]{\begin{adjustwidth}{0.25cm}{0.25cm}\textbf{Afirmação #1: }}{\end{adjustwidth}}
\newcommand{\vect}[1]{\mathbf{#1}}
\newcommand{\Bij}{\operatorname{Bij}} % default value = empty, for first parameter
\newcommand{\sgn}{\operatorname{sgn}}
\newenvironment{generic}[2]{\begin{adjustwidth}{0.25cm}{0.25cm}\textbf{#1 #2: }}{\end{adjustwidth}}
\newcommand{\inner}[2]{\langle #1, #2 \rangle}
\newcommand{\emptyinner}[0]{\langle , \rangle}
\newcommand{\R}{\mathbb{R}}
\newcommand{\K}{\mathbb{K}}
\newcommand{\N}{\mathbb{N}}

% NOT DOWNLOADED YET
% \usepackage{algorithm}
% \usepackage{algpseudocode}

\begin{document}
\section{Introdução}
Esse problema foi retirado de \url{https://cses.fi/problemset/task/1636}, da plataforma de programação competitiva CSES. 


\section{Código em C++}

\begin{code}[C++]
#include<bits/stdc++.h>
using namespace std;
	
/*
Construct solutions from array: 
* every ordered solution can be represented as a n-uple (k_1, ... k_n)
	where k_i is the amount of times c[i] appears in the sum
* however, code this solution BOTTOM UP
*/
	
	
void solve(){
	const int MOD = 1e9 + 7;
	int n, x;
	cin >> n >> x;
	int c[n];
	for (int i = 0; i < n; i++){
	cin >> c[i];
	}
	int dp[x + 1] = {0};
	dp[0] = 1;
	
	for (int i = 0; i < n; i++){
	for (int s = c[i]; s <= x; s++){
		dp[s] = (dp[s] + dp[s - c[i]]) % MOD;
	}
	}
	cout << dp[x];
}	
\end{code}


\section{Demonstração}
\begin{lemma}
	Cada solução ordenada \((s_1, s_2, \dotsc, s_k)\) é unicamente representada por um vetor \((a_1, \dotsc, a_n)\), em que \(a_i\) é a quantidade de vezes que a moeda \(c[i]\) foi utilizada.
\end{lemma}

\begin{lemma}
	Denote por \(C(i, j)\) a quantidade de combinações \((a_1, \dotsc, a_n)\) com soma \(j\) que usam apenas as moedas \(c[0], \dotsc, c[i]\), isto é, com \(a_{i+1} = \dotsc = a_n = 0\), e (extendendo essa definição) deixe \(C(-1, j)\) denotar a quantidade de combinações vazias com soma \(j\).

	Então, após a \(i\)-ésima iteração, \(dp[j] = C(i, j)\). 
\end{lemma}
\begin{proofenv}
	Primeiro, provamos a seguinte sub-invariante de loop:
	\begin{lemma}
		\textbf{Assuma} a hipótese de indução do lema principal, isto é, que temos inicialmente \(dp[j] = C(i, j)\) para todo \(0 \leq j \leq x\). Então, durante a \((i+1)\)-ésima iteração do loop externo, o seguinte é válido: para todo \(0 \leq w \leq x\) temos que antes da iteração \(w = w_0\) do loop interno, \(dp[j]\) representa quantas combinações de soma \(j\) há usando apenas as moedas \(c[0], \dotsc, c[i+1]\) para todo \(0 \leq j < w_0\).
	\end{lemma}
	\begin{proofenv}
		Antes da primeira iteração \(w = c[i+1]\) do loop interno, temos que \(j < w \implies j < c[i+1]\), o que implica evidentemente que não há combinação com essa soma que use \(c[i+1]\). Dessa forma, temos que \(C(i + 1, j) = C(i, j) = dp[j]\), pela hipótse indutiva.

		Por fim, suponha válido para \(w = w_0\). Toda combinação que usa as moedas \(c[0], \dotsc, c[i+1]\) é de duas formas: 
		\begin{itemize}
			\item \(a_{i+1} = 0\) (\(C(i, w_0 + 1)\) possibilidades)
			\item \(a_{i+1} \geq 1\), e portanto a combinação obtida retirando uma repetição de \(c[i+1]\) é uma combinação válida de soma \(w_0 + 1 - c[i+1]\) usando as moedas \(c[0], \dotsc, c[i+1]\). Isso totaliza \(C(i + 1, w_0 + 1 - c[i+1]\).
		\end{itemize}
		Assim,
		\[C(i + 1, w_0 + 1) = C(i, w_0 + 1) + C(i + 1, w_0 + 1 - c[i+1])\]
		Uma vez que \(w_0 + 1 - c[i+1] \leq w_0\), pela hipótese indutiva desse sub-lema temos que nesse instante \(C(i+1, w_0 + 1 - c[i+1] = dp[w_0 + 1 - c[i+1]\). Por outro lado, temos que \(dp[w_0 + 1]\) não foi alterada até agora nesse loop interno, e portanto pela hipótese indutiva do lema principal \(dp[w_0 + 1] = C(i, w_0 + 1)\). Portanto, a ação do algoritmo é 
		\[dp[w_0 + 1] \gets (dp[w_0 + 1] + dp[w_0 + 1 - c[i+1] = C(i, w_0 + 1) + C(i + 1, w_0 + 1 - c[i+1])\]
		garantindo a manutenção da invariante descrita nesse sub-lema.
	\end{proofenv}
	Dessa forma, a invariante principal (i.e, do loop externo) segue com um simples argumento indutivo. A inicialização de \(dp[0] \gets 1\) e \(dp[j] \gets 0\) para todo \(j > 0\) garante que \(dp[j] = C(-1, j)\) para todo \(0 \leq j \leq x\). 

	Assim, aplicando o lema anterior para \(i = -1\), demonstramos o caso base: a validade ao final da iteração \(i = 0\). Analogamente, o passo indutivo também segue da aplicação do lema anterior.

\end{proofenv}

\end{document}
