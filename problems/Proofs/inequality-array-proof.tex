\documentclass{article}
\usepackage[utf8]{inputenc}

% imports 
%Prefer 
%\iff, \implies and \impliedby

%Bibliography stuff
%For adding hyperlinks
\usepackage[hyphens]{url} % avoids too large urls by allowing line breaking at hyphens
% (best loaded BEFORE hyperref)

\usepackage{parskip} %avoid margins in the beginning of paragraphs
%Math symbols
\usepackage{amsmath}
\usepackage{amsthm}
\usepackage{amssymb}
%Paper geometry
\usepackage{geometry}
\geometry{a4paper, margin = 1cm}

% CODE LISTING
\usepackage{listings}
\lstnewenvironment{code}[1][C++]
{\lstset{language=C++,numbers=left,numberstyle=\tiny}
  \minipage{\dimexpr\textwidth-2em}
}
{\endminipage}

%For adding images
\usepackage{graphicx}
%Defining floor symbols
\usepackage{mathtools}
\DeclarePairedDelimiter\ceil{\lceil}{\rceil}
\DeclarePairedDelimiter\floor{\lfloor}{\rfloor}
%Fixing parenthesis sizes
% \usepackage{physics} %use \qty before parenthesis % NOT DOWNLOADED YET
%Adjustable margins
\def\changemargin#1#2{\list{}{\rightmargin#2\leftmargin#1}\item[]}
\let\endchangemargin=\endlist %allows you to create an environment \begin{changemargin}{x cm}{y cm} \end{changemargin} - where x is the adjust to left margin and y to the right margin 
\usepackage{changepage}
\newtheorem{theorem}{Teorema}[section] %Delimit "the numbered theorem" environment
%the [section] counter establishes that counting resets every section, i.e second theorem in section 7 will be 7.2. 
\newtheorem{lemma}{Lema}[section]
\newtheorem{corollary}{Corolário}[theorem]
\newtheorem{definition}{Definição}[section]
\newtheorem{example}{Exemplo}[section]
\usepackage{hyperref} %To be able to label lemmas and theorems with \label{} and automate the process of referring to them with \cref{} 
%Names for referrencing
\usepackage{cleveref}
\crefname{lemma}{Lema}{Lemas}
\crefname{theorem}{Teorema}{Teoremas}
\crefname{corollary}{Corolário}{Corolários}
\crefname{definition}{Definição}{Definições}
\crefname{example}{Exemplo}{Exemplos}
\crefname{section}{Seção}{Seções}
\crefname{subsection}{Subseção}{Subseções}

%Proof, solution, claim and case environments
\newenvironment{proofenv}{\begin{adjustwidth}{0.25cm}{0.25cm}\begin{proof}[Prova]}{\end{proof}\end{adjustwidth}}
\newenvironment{solutionenv}{\begin{adjustwidth}{0.25cm}{0.25cm}\begin{proof}[Solução]}{\end{proof}\end{adjustwidth}}
\newenvironment{case}[1][1]{\begin{adjustwidth}{0.25cm}{0.25cm}\textbf{Caso #1: }}{\end{adjustwidth}}
%[1] is the first paramater (case num), [1] is its default value
\newenvironment{claim}[1][1]{\begin{adjustwidth}{0.25cm}{0.25cm}\textbf{Afirmação #1: }}{\end{adjustwidth}}
\newcommand{\vect}[1]{\mathbf{#1}}
\newcommand{\Bij}{\operatorname{Bij}} % default value = empty, for first parameter
\newcommand{\sgn}{\operatorname{sgn}}
\newenvironment{generic}[2]{\begin{adjustwidth}{0.25cm}{0.25cm}\textbf{#1 #2: }}{\end{adjustwidth}}
\newcommand{\inner}[2]{\langle #1, #2 \rangle}
\newcommand{\emptyinner}[0]{\langle , \rangle}
\newcommand{\R}{\mathbb{R}}
\newcommand{\K}{\mathbb{K}}
\newcommand{\N}{\mathbb{N}}

% NOT DOWNLOADED YET
% \usepackage{algorithm}
% \usepackage{algpseudocode}

\begin{document}
\section{Enunciado}
O problema é o \textbf{1703F}, do contest 806 (Div4), da plataforma CodeForces, disponível em \url{https://codeforces.com/contest/1703/problem/F}

\section{Solução (em C++)}

\begin{code}[C++]
#include <bits/stdc++.h>
using namespace std;
    
template <typename T> 
void ps(T a[], int n){
    for (int i = 1; i < n; i++){
        a[i] += a[i-1];
    }
}
  
void solve(){
    int n;
    cin >> n;
    int a[n+1]; // to solve 1-indexing we prepend 0 to the array
    a[0] = 0;
    for (int i = 1; i < n + 1; i++) cin >> a[i];
    
    vector<int> s;
    for (int i = 0; i < n + 1; i++){
        if (a[i] < i) s.emplace_back(i);
    } // guaranteed 0 not in s 
    
    /*calculating #(s intersection [0, a[j])) fast with prefix sums*/
    long long b[n + 1] = {0ll}; // avoid OVERFLOW
    for (int i : s) ++b[i];
    ps(b, n + 1);
    // then #(s \cap [0, a[j])) = b[a[j] - 1]
    
    long long ans = 0; // avoid OVERFLOW
    for (int i: s){
        if (a[i] > 0){
              ans += b[a[i] - 1];
        } 
        // else if a[i] == 0 then there is no j in S with j < a[i]
    }
    cout << ans << '\n';
}
\end{code}  



\section{Demonstração}
Defina por \(S\) o conjunto de índices \(S = \{i: a[i] < i\}\). Então temos que \((i, j)\) é um par válido, isto é, que satisfaz \(a[i] < i < a[j] < j\), se e só se \(i \in S, j \in S\) e \(i < a[j]\).

Portanto, para cada \(j \in S\) temos que existem exatamente \(|\{i \in S: i < a[j]| = |S \cap \{1, \dotsc, a[j] - 1\}|\) pares válidos de índices em que o maior índice é \(j\). Segue que a resposta final é 
\[\operatorname{ans} = \sum_{j \in S}|S \cap \{1, \dotsc, a[j] - 1\}|\]

Considere a função característica \(f_S: \{1, \dotsc, n\} \to \{0,1\}\) dada por \(f_S^{-1}(0) = \{1, \dotsc, n\} \setminus S\) e \(f_S^{-1}(1) = S\). Assim, temos que
\[|S \cap \{1, \dotsc, a[j] - 1\}| = \sum_{1 \leq i < a[j]}f_S(i)\]
que é exatamente \(\operatorname{prefix\_sum}(b)[a[j] - 1]\) se \(a[j] > 1\), ou do contrário \(0\) (devido à soma vazia), em que \(b\) é o vetor caracerístico de \(S\). Denotemos essa função definida por partes por \(\sigma_b\).  

Portanto, podemos computar \(b\) em \(O(n)\), podemos realizar a \(\operatorname{prefix\_sum}\) desse vetor também em \(O(n)\), e por fim, com todas essas informações, a resposta final 
\[\operatorname{ans} = \sum_{j \in S}|S \cap \{1 \dotsc, a[j] - 1\}| = \sum_{1 \leq j \leq n}b[j]\sigma_b(a[j] - 1)\]
pode ser computada em \(O(n)\).

Assim, mostra-se que essa é uma solução correta e com complexidade linear \(O(n)\).
\end{document}
