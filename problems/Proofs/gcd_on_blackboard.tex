\documentclass{article}
\usepackage[utf8]{inputenc}

% imports 
%Prefer 
%\iff, \implies and \impliedby

%Bibliography stuff
%For adding hyperlinks
\usepackage[hyphens]{url} % avoids too large urls by allowing line breaking at hyphens
% (best loaded BEFORE hyperref)

\usepackage{parskip} %avoid margins in the beginning of paragraphs
%Math symbols
\usepackage{amsmath}
\usepackage{amsthm}
\usepackage{amssymb}
%Paper geometry
\usepackage{geometry}
\geometry{a4paper, margin = 1cm}

% CODE LISTING
\usepackage{listings}
\lstnewenvironment{code}[1][C++]
{\lstset{language=C++,numbers=left,numberstyle=\tiny}
  \minipage{\dimexpr\textwidth-2em}
}
{\endminipage}

%For adding images
\usepackage{graphicx}
%Defining floor symbols
\usepackage{mathtools}
\DeclarePairedDelimiter\ceil{\lceil}{\rceil}
\DeclarePairedDelimiter\floor{\lfloor}{\rfloor}
%Fixing parenthesis sizes
% \usepackage{physics} %use \qty before parenthesis % NOT DOWNLOADED YET
%Adjustable margins
\def\changemargin#1#2{\list{}{\rightmargin#2\leftmargin#1}\item[]}
\let\endchangemargin=\endlist %allows you to create an environment \begin{changemargin}{x cm}{y cm} \end{changemargin} - where x is the adjust to left margin and y to the right margin 
\usepackage{changepage}
\newtheorem{theorem}{Teorema}[section] %Delimit "the numbered theorem" environment
%the [section] counter establishes that counting resets every section, i.e second theorem in section 7 will be 7.2. 
\newtheorem{lemma}{Lema}[section]
\newtheorem{corollary}{Corolário}[theorem]
\newtheorem{definition}{Definição}[section]
\newtheorem{example}{Exemplo}[section]
\usepackage{hyperref} %To be able to label lemmas and theorems with \label{} and automate the process of referring to them with \cref{} 
%Names for referrencing
\usepackage{cleveref}
\crefname{lemma}{Lema}{Lemas}
\crefname{theorem}{Teorema}{Teoremas}
\crefname{corollary}{Corolário}{Corolários}
\crefname{definition}{Definição}{Definições}
\crefname{example}{Exemplo}{Exemplos}
\crefname{section}{Seção}{Seções}
\crefname{subsection}{Subseção}{Subseções}

%Proof, solution, claim and case environments
\newenvironment{proofenv}{\begin{adjustwidth}{0.25cm}{0.25cm}\begin{proof}[Prova]}{\end{proof}\end{adjustwidth}}
\newenvironment{solutionenv}{\begin{adjustwidth}{0.25cm}{0.25cm}\begin{proof}[Solução]}{\end{proof}\end{adjustwidth}}
\newenvironment{case}[1][1]{\begin{adjustwidth}{0.25cm}{0.25cm}\textbf{Caso #1: }}{\end{adjustwidth}}
%[1] is the first paramater (case num), [1] is its default value
\newenvironment{claim}[1][1]{\begin{adjustwidth}{0.25cm}{0.25cm}\textbf{Afirmação #1: }}{\end{adjustwidth}}
\newcommand{\vect}[1]{\mathbf{#1}}
\newcommand{\Bij}{\operatorname{Bij}} % default value = empty, for first parameter
\newcommand{\sgn}{\operatorname{sgn}}
\newenvironment{generic}[2]{\begin{adjustwidth}{0.25cm}{0.25cm}\textbf{#1 #2: }}{\end{adjustwidth}}
\newcommand{\inner}[2]{\langle #1, #2 \rangle}
\newcommand{\emptyinner}[0]{\langle , \rangle}
\newcommand{\R}{\mathbb{R}}
\newcommand{\K}{\mathbb{K}}
\newcommand{\N}{\mathbb{N}}

% NOT DOWNLOADED YET
% \usepackage{algorithm}
% \usepackage{algpseudocode}

\begin{document}
\section{Introdução}
Esse problema é o problema C do \emph{AtCoder Begginer Contest 125}, da plataforma de programação competitiva AtCoder. Disponível em: \url{https://atcoder.jp/contests/abc125/tasks/abc125_c}.

\section{Solução: código em C++}\label{code-section}

\begin{code}[C++]
#include <bits/stdc++.h>
using namespace std;
    
/*
left[i] = gcd(a[0], ..., a[i-1])
right[i] = gcd(a[i + 1], ..., a[n - 1])
*/
    
int gcd(int a, int b){
    int m, M;
    while (a != 0 && b != 0){
        m = min(a,b); 
        M = max(a,b);
            b = M % m;
            a = m;
    }
        return max(a, b);
}
    
void solve(){
    int n;
    cin >> n;
    int a[n];
    for (int i = 0; i < n; i++){
        cin >> a[i];
    }
    int left[n], right[n];
    left[0] = 0; right[n - 1] = 0;
    for (int i = 1; i < n; i++){
        left[i] = gcd(left[i - 1], a[i - 1]);
    }
    for (int i = n - 2; i >= 0; i--){
        right[i] = gcd(right[i + 1], a[i + 1]);
    }
    
    int ans = 0;
    for (int i = 0; i < n; i++){
        ans = max(ans, gcd(left[i], right[i]));
    }
    cout << ans;
}
\end{code}

\section{Demonstração}

\subsection{Lemas importantes sobre \emph{gcd}}
\begin{theorem}\label{associativity}
    A função \(\gcd: \N \to \N\) é associativa:
    \[\gcd(a, \gcd(b, c)) = \gcd(\gcd(a, b), c)\]
    Ademais, ela é simétrica em relação a seus argumentos: \(\gcd(a, b) = \gcd(b, a)\).
\end{theorem}
\begin{proof}[Prova]
    Pelo Teorema Fundamental da Aritmética, temos
    \begin{align*}
        a = \prod_{1 \leq i \leq n}p_i^{\alpha_i} \\
        b = \prod_{1 \leq i \leq n}p_i^{\beta_i} \\
        c = \prod_{1 \leq i \leq n}p_i^{\gamma_i}
    \end{align*}
    em que \(p_1 < \dotsc, < p_n\) são os \(n\)-primeiros primos e um dos três expoentes \(\alpha_n, \beta_n, \gamma_n\) é diferente de \(0\). Assim o lema segue da \textbf{associatividade da função \(\boldsymbol{\min}\)}: 
    \[\gcd(a, \gcd(b, c)) = \prod_{1 \leq i \leq n}p_i^{\min(\alpha_i, \min(\beta_i, \gamma_i))} = \prod_{1 \leq i \leq n}p_i^{\min(\min(\alpha_i, \beta_i), \gamma_i)} = \gcd(\gcd(a, b), c)\]
    A associatividade da função \(\min\), por sua vez, pode ser provada analisando caso a caso dentre as possíveis ordenação de três números, e portanto sua demonstração é omitida aqui.
\end{proof}


Assim, podemos definir, para todo \(k \geq 2\), a função \(\gcd_k: \N^k \to \N\) definida recursivamente por 
\begin{definition}\label{n-ary-gcd}
    \[\gcd_{k+1}(a_1, \dotsc, a_{k+1}) = \gcd(\gcd_{k}(a_1, \dotsc, a_k), a_{k+1})\]    
\end{definition}
a qual denotaremos apenas por \(\gcd\), ficando a quantidade de parâmetros implícita.

Como corolário, vale a seguinte generalização da associatividade:
\begin{corollary}\label{complete-associativity}
    Para todos \(1 \leq i < n\):
    \[\gcd_n(a_1, \dotsc, a_n) = \gcd(\gcd_{i}(a_1, \dotsc, a_{i}),\ \gcd_{n-i}(a_{i+1}, \dotsc, a_n))\]
    em que definimos \(\gcd_1(a) = a\).
\end{corollary}
\begin{proof}[Prova]
    Provamos por indução sobre \(n - i\). O caso \(n - i = 1\) segue da própria \textbf{definição} de \(\gcd_n\). Além disso,
    \begin{align*}
        \gcd(
            \gcd_i(a_1, \dotsc, a_i), 
            \gcd_{n - i}(a_{i+1}, \dotsc, a_{n-i})
        ) 
        &\overset{\text{\cref{n-ary-gcd}}}{=} 
        %%
        \gcd(
            \gcd(\gcd_{i-1}(a_1, \dotsc, a_{i-1}), a_i), \
            \gcd_{n-i}(a_{i+1}, \dotsc, a_n)
        )  
        \\
        &\overset{\text{\cref{associativity}}}{=} 
        \gcd(
            \gcd_{i-1}(a_1, \dotsc, a_{i-1}), \
            \gcd(
                a_i, 
                \gcd_{n-i}(a_{i+1}, \dotsc, a_n)
            )
        )
        \\
        &=
        \gcd(
            \gcd_{i-1}(a_1, \dotsc, a_{i-1}),\
            \gcd(
                \gcd_{n-i}(a_{i+1}, \dotsc, a_{n}),\
                a_i
            )
        )
        \\ 
        &\overset{\text{\cref{n-ary-gcd}}}{=} 
        \gcd(
            \gcd_{i-1}(a_1, \dotsc, a_{i-1}),\
            \gcd_{n-i + 1}(a_{i+1}, \dotsc, a_n, a_i)
        )
    \end{align*} 
    %% TODO prove that the order does not matter
    porém, como a ordem dos argumentos de \(\gcd_k\) não importa, para qualquer \(k\) natural, então a expressão acima é simplesmente \(\gcd(
        \gcd_{i-1}(a_1, \dotsc, a_{i-1}),\ 
        \gcd_{n-i+1}(a_i,\dotsc,a_n)
    )\). Ou seja, usando a hipótese de indução
    \[\gcd_n(a_1, \dotsc, a_n) = \gcd(
        \gcd_i(a_1, \dotsc, a_i), \
        \gcd_{n-i}(a_{i+1}, \dotsc, a_n)
    ) 
    = 
    \gcd(
        \gcd_{i-1}(a_1, \dotsc, a_{i-1}),\ 
        \gcd_{n-i+1}(a_i,\dotsc,a_n)
    )\]     
    provando o corolário por indução.
\end{proof}

\begin{lemma}\label{gcd-remainder}
    Para todos \(a, b \in \N\), se \(b \neq 0\) e \(a \equiv r \mod b\) então
    \[\gcd(a, b) = \gcd(r, b)\]
\end{lemma}

\begin{theorem}\label{gcd-algorithm-efficiency}
    O seguinte algoritmo para calcular \(\gcd\) de dois números
    
    \begin{code}[C++]
        int gcd(int a, int b){
            while (a != 0 and b != 0){
                int m = min(a,b);
                int M = max(a, b);
                a = M % m;
                b = m;
            }
        return max(a, b);
        }        
    \end{code}

    termina com o resultado correto em \(O(\log \min(a, b))\).
\end{theorem}
\begin{proof}[Prova: \textbf{Complexidade}]
    Seja \(a_i, b_i\) os valores das variáveis após a 
    \(i\)-ésima iteração, e \(a_0 = a, b_0 = b\) seus valores iniciais.

    A partir da primeira iteração temos como invariante de loop
    que \(a_i \leq b_i\), uma vez que \(a_i \gets M \mod m \leq m\) e \(b_i \gets m\). Assim, podemos supor sem perda de generalidade que \(a \leq b\) inicialmente (do contrário, basta \(1\) iteração para valer tal desigualdade).
    
    Após uma iteração temos \(a_{i+1} = b_i \bmod a_i\) e \(b_{i+1} = a_i\), e logo, após duas iterações temos que
    \begin{align}
        a_{i+2} &= a_i \bmod (b_i \bmod a_i) \\
        b_{i+2} &= b_i \bmod a_i
    \end{align}
    Seja \(b_i = q_ia_i + r_i\). Então temos dois casos:
    \begin{case}[1]
        Se \(r_i \leq \frac{a_i}{2}\) então 
        \[a_{i+2} = a_i \mod r_i < r_i \leq \frac{a_i}{2}\]
    \end{case}
    \begin{case}[2]
        Se \(r_i > \frac{a_i}{2}\), então 
        \[a_{i+2} = a_i \mod r_i = a_i - r_i < a_i - \frac{a_i}{2} = \frac{a_i}{2}\]        
    \end{case}
    De qualquer forma temos \(a_{i+2} \leq 2^{-1}a_i\). Logo, por indução
    \[a_{2i} \leq 2^{-i}a_0\]

    Como \(a_{2i}\) é um inteiro não-negativo, temos que \(i > \log_2(a_0)\) implica que \(a_{2i} < 1\), e portanto \(a_{2i} = 0\), o que faz que o algoritmo termine. Como supomos sem perda de generalidade \(a_0  \leq b_0\), isso mostra que o algoritmo termina em no máximo \(2(\log_2(a_0) + 1) = 2(\log_2(\min(a, b)) + 1)\) iterações, isto é, possui complexidade \(O(\log \min(a, b))\). 
\end{proof}
\begin{proof}[Prova: \textbf{Corretude}]
    Precisamos mostrar apenas a seguinte invariante de loop: após \(i\)-ésima iteração
    \[\gcd(a_i, b_i) = \gcd(a, b)\]

    O caso inicial \(i = 0\) segue da definição dos termos iniciais da sequência, pois \(a_0 := a\) e \(b_0 := b\). Suponha, então, que existe \(i > 0\) tal que a invariante de loop segue válida para todo \(0 \leq j \leq i\). Conforme mostramos na parte anterior, e pela condição do loop while, temos \(0 < a_i \leq b_i\), e portanto
    \[\gcd(a_{i+1}, b_{i+1}) = \gcd(b_i \bmod a_i, a_i) = \gcd(b_i, a_i) = \gcd(a_i, b_i)\]
    pelo \cref{gcd-remainder}. Pela parte anterior da demonstração, após a iteração final \(i_0\) temos \(a_{i_0} = 0\) ou \(b_{i_0} = 0\), e portanto
    \[\gcd(a, b) \overset{\text{invariante}}{=} \gcd(a_{i_0}, b_{i_0}) = \max(a_{i_0}, b_{i_0})\]
    provando a corretude do algoritmo.    
\end{proof}

%%%%%%%%%%%%%%%%%%%%%%%%%%%%%%%%%%%%%%%%%%%

\subsection{Demonstrando o algoritmo}

\begin{theorem}\label{main-result-correctness}
    O algoritmo da \cref{code-section} fornece a resposta correta em \(O(n \log M)\), onde \(M = \max(a[0], \dotsc, a[n-1])\).
\end{theorem}

\begin{proof}[Prova]
    Temos que a resposta que deve ser fornecida, dado um vetor \(a[]\) de \(n\) inteiros positivos, é 
    \[
        S = \max_{b \in \mathbb{Z}_{\geq 0}}\max_{1 \leq i \leq n}\gcd(a[0], \dotsc, a[i-1], b, a[i+1], \dotsc, a[n-1])    
    \]
    Porém, temos que para todo \(b \geq 0\) inteiro:
    \begin{align*}
        \gcd(
            a[0], \dotsc, a[i-1], b, a[i+1], \dotsc, a[n-1]
        )  
        &= \gcd(a[0], \dotsc, a[i-1], a[i+1], \dotsc, a[n-1], b)
        \\
        &\overset{\text{\cref{n-ary-gcd}}}{=} 
        \gcd(
            \gcd(a[0], \dotsc, a[i-1], a[i+1], \dotsc, a[n-1]),\ 
            b
        ) 
        \\
        &\leq 
        \gcd(a[0], \dotsc, a[i-1], a[i+1], \dotsc, a[n-1])
        \\ 
        &\overset{\text{\cref{complete-associativity}}}{=}
        \gcd(
            \gcd(a[0], \dotsc, a[i-1]),\
            \gcd(a[i+1], \dotsc, a[n-1])
        )
    \end{align*}
    com \textbf{caso de igualdade quando, por exemplo}, \(b = \gcd(a[0], \dotsc, a[i-1], a[i+1], \dotsc, a[n-1])\). 

    Assim, podemos definir os seguintes vetores \(L[n], \ R[n]\):
    \begin{align*}
        L[0] &= 0 \\
        L[i + 1] &= \gcd(L[i], a[i]) \\
        \\
        R[n - 1] &= 0 \\
        R[i - 1] &= \gcd(R[i], a[i])
    \end{align*}
    e pela associatividade (\cref{associativity}) teremos
    \begin{align*}
        L[i] = \gcd(a[0], \dotsc, a[i - 1]) \\
        R[i] = \gcd(a[n - 1], \dotsc, a[n - i])
    \end{align*}

    Ademais, novamente pelo \cref{associativity}, podemos calcular ambos os vetor em \(O(n) \cdot O(\gcd) = O(n\log M)\) onde \(M = \max(a[0], \dotsc, a[n-1])\). 

    Por fim, teremos que 
    \begin{align*}
        S &= \max_{0 < i < n - 1}
        \gcd(
            \gcd(a[0], \dotsc, a[i-1]),\
            \gcd(a[i+1], \dotsc, a[n-1])
        ) \\
        &=
        \max_{0 < i < n - 1}\gcd(L[i], R[n-i])
    \end{align*}
    o que novamente pode ser calculado em \(O(n \log M)\).

    Portanto, o código da \cref{code-section}, que é exatamente a implementação do algoritmo descrito, fornece uma solução correta em \(O(n \log M)\).
\end{proof}
 
\end{document}