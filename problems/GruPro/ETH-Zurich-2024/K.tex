\documentclass{article}
\usepackage{amsmath, amsthm, amssymb, amsfonts}
\author{Igor Borja}
\date{}

\title{Problema K - ETH Zurich Competitive Programming Contest Spring 2024}

\begin{document}
    \maketitle

    \section{Análise}

    \textbf{Note que nenhuma moeda é paga caso ele seja sorteado}.

    Para que Oli gaste exatamente \(q > 0\) moedas, é necessário que
    \begin{itemize}
        \item  \((q - 1) \cdot n\) pessoas não sejam sorteadas, (\((q - 1)\) ciclos completos), para que ele gaste \(q - 1\) moedas
        \item As \(k - 1\) pessoas na sua frente, e ele próprio, não sejam sorteadas (portanto, \(k\) pessoas)
        \item Alguma das \(n - k\) pessoas atrás dele seja sorteada, ou ninguém seja sorteado nesse ciclo e uma das \(k - 1\) pessoas na frente dele seja sorteada (ou ele próprio) no ciclo seguinte. Ou seja, (pensando em posições \textbf{zero-indexadas} - sendo que Oli está na \((k - 1)\)-ésima posição) existe um \(0 \leq j \leq n - 1\) tal que \(k, \dotsc, (k + j - 1) \mod n\) não são sorteados e \(k + j \mod n\) é sorteado, depois das duas condições anteriores.
    \end{itemize}

    Ou seja, pondo \(p = \dfrac{a}{b}\) temos que a probabilidade de ele gastar exatamente \(q\) moedas é:

    \begin{align*}
        P(X = q)
        &= \left(1 - p\right)^{(q - 1) \cdot n + k}\sum_{j = 0}^{n - 1}\left(1 - p\right)^{j}p
        \\
        &= p\left(1 - p\right)^{(q - 1)\cdot n + k} \dfrac{1 - \left(1 - p\right)^{n}}{1 - \left(1 - p\right)}
        = \left(1 - p\right)^{(q - 1)\cdot n + k}(1 - (1 - p)^n)
    \end{align*}

    Portanto, o valor esperado é dado por:
    \begin{align*}
        \mathbb{E}[X] &= \sum_{q > 0}q P(q) 
        = \left(1 - p\right)^{k}(1 - (1 - p)^n)\sum_{q > 0}q\left(1 - p\right)^{(q - 1)n}
    \end{align*}

    Porém, é fácil mostrar que, para toda P.G de razão \(x\) com \(|x| < 1\):
    \begin{align*}
        \sum_{k \geq 0}kx^{k - 1} = \dfrac{d}{dx}\sum_{k \geq 0}x^k = \dfrac{d}{dx}\dfrac{1}{1 - x} = \dfrac{1}{(1 - x)^2}
    \end{align*}
    e portanto:

    \begin{align*}
        \mathbb{E}[X]
        &= (1 - p)^{k}(1 - (1 - p)^n)\dfrac{1}{\left(1 - \left(1 - p\right)^n\right)^2}
        = \dfrac{(1 - p)^{k}}{(1 - (1 - p)^n)}
        \\
        &= \dfrac{\left(\dfrac{b - a}{b}\right)^{k}}{1 - \left(\dfrac{b - a}{b}\right)^n}
        = \dfrac{(b - a)^{k}b^{n - k}}{b^n - (b - a)^n}
    \end{align*}

    \section{Implementação}

    \begin{verbatim}
#include <bits/stdc++.h>
using namespace std;
    
#define i64 int64_t
const i64 MOD = (i64)1e9 + 7;
    
i64 bexp(i64 a, i64 ex){
    if (ex == 0) {
        return 1;
    } else {
        i64 b = bexp(a, ex / 2);
        if (ex % 2 == 0){
            return (b * b) % MOD;
        } else {
            return (a * ((b * b) % MOD)) % MOD;
        }
    }
}
    
    
int main(){
    i64 n, k, a, b;
    cin >> n >> k >> a >> b;
    
    i64 num = (bexp(b - a, k) * bexp(b, n - k)) % MOD;
    i64 denum = (bexp(b, n) + MOD - bexp(b - a, n)) % MOD;
    cout << (num * bexp(denum, MOD - 2)) % MOD << endl;
}
    \end{verbatim}
\end{document}