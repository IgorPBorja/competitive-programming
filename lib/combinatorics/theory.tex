\subsection{Fórmula de Legendre}

Para primo $p$, $v_p(n!) = \sum_{k=1}^{\infty} \left\lfloor \frac{n}{p^k} \right\rfloor$, e
$v_p\left(\binom{n}{k}\right) = v_p(n!) - v_p(k!) - v_p((n-k)!)$.

Para $m = \prod p_i^{e_i}$ composto, $v_m(n!) = \min_i \left\lfloor \dfrac{v_{p_i}(n!)}{e_i} \right\rfloor$.

\subsection{Teorema de Lucas}

Para $p$ primo e $n = n_k p^k + \dotsc + n_1 p + n_0$ e $m = m_k p^k + \dotsc + m_1 p + m_0$,
\[\binom{n}{m} \equiv \prod_{i=0}^{k}\binom{n_i}{m_i} \mod p\]

\subsection{Teorema de Sprague-Grundy e game-theory}

Nim: ganha se e só se o XOR dos números da pilha é maior que $0$: $a_1 \oplus \dotsc \oplus a_n \neq 0$.

Estados terminais tem número de Grundy $n(v) = 0$ e não-terminais tem número de Grundy $n(v) = \operatorname{mex}\limits_{v \to u}n(u)$. Assim,
um estado é \textbf{vencedor} (em um jogo em que quem não tem movimentos perde) se e só se $n(v) \neq 0$.
Um único jogo é equivalente a uma \textbf{pilha de Nim com (limitados) acréscimos}.

Assim, se temos dois "sub-jogos" independentes $S_1$ e $S_2$ então seu número de Grundy é o XOR dos números de Grundy individuais, $n(S_1 + S_2) = n(S_1) \oplus n(S_2)$.

\subsection{Stars and bars}

Número de soluções inteiras para $x_1 + x_2 + \dotsc + x_k = n$ com $x_i \geq 1$ para todo $i$ é $\binom{n - 1}{k - 1}$
($n-1$ posições - aquelas entre $j$ e $j+1$ para $j=1\dotsc n-1$ - para $k-1$ barras separadoras, sendo que cada posição só pode ser ocupada por no máximo uma barra)

Número de soluções inteiras para $x_1 + x_2 + \dotsc + x_k = n$ com $x_i \geq 0$ para todo $i$ é $\binom {n + k - 1}{k - 1}$
($n$ unidades, $k-1$ barras separadoras, qualquer permutação é válida).
