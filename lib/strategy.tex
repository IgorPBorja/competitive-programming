\pagebreak

\section{Strategy}

\subsection{Debugging:}

\begin{itemize}
	\item \textbf{Nem sempre o bug está na parte mais complexa do código: leia tudo, até o template}
	\item Verifique o valor do MOD, do INF e das constantes, dos tamanhos dos arrays.
	\item Criar corner cases ($n=0,1$, casos especiais que forçam alguma coisa, etc) e casos pequenos
	\item Colocar asserts para detectar índices out-of-bounds ou outros problemas
	\item \textbf{Minimizar estado global, separar lógica em funções menores}
\end{itemize}

\subsection{Início (primeira 1h30)}

\begin{itemize}
	\item Ler questões
	\item Passar no máximo 10-15 min por questão
	\item Implementar apenas as fáceis
	\begin{itemize}
		\item Priorizar primeiro aquelas que são simples e rápidas de implementar
		(em detrimento por ex. de uma aplicação trivial de um algoritmo longo, como uma SegTree ou fluxo)
	\end{itemize}
	\item Fase mais individual
	\item Priorizar para implementação aquela pessoa que tem a ideia mais clara da solução
\end{itemize}

\subsection{Meio}

\begin{itemize}
	\item Ler o restante da prova
	\item Olhar os standings para selecionar a próxima questão
	\item Analisar mais profundamente no papel para conseguir passar as médias/difíceis
	\item Evitar ficar três pessoas em uma mesma questão
\end{itemize}

\subsection{Final (última 1h30)}

\begin{itemize}
	\item Foco do time todo em uma ou duas questões mais prováveis de passarem
\end{itemize}

\pagebreak